% `advanced_example.tex', an advanced example employing the AIAA class
% plus other third-party LaTeX packages.
%
% For a bare-bones usage, see `template.tex'.
%
% Typical processing for PostScript (PS) output:
%
%  latex advanced_example
%  bibtex advanced_example  (bibliography)
%  makeindex -s nomencl.ist -o advanced_example.gls advanced_example.glo
%                            (nomenclature)
%  latex advanced_example   (repeat as needed to resolve references)
%
%  xdvi advanced_example    (onscreen draft display)
%  dvips advanced_example   (postscript)
%  gv advanced_example.ps   (onscreen display)
%  lpr advanced_example.ps  (hardcopy)
%
% With the above, only Encapsulated PostScript (EPS) images can be used.
%
%
%  pdflatex advanced_example
%  bibtex advanced_example    (bibliography)
%  makeindex -s nomencl.ist -o advanced_example.gls advanced_example.glo
%                              (nomenclature)
%  pdflatex advanced_example  (repeat as needed to resolve references)
%
%  acroread advanced_example.pdf  (onscreen display)
%
% If you have EPS figures, you will need to use the epstopdf script
% to convert them to PDF because PDF is a limmited subset of EPS.
% pdflatex accepts a variety of other image formats such as JPG, TIFF,
% PNG, and so forth -- check the documentation for your version.
%
% If you do *not* specify suffixes when using the graphicx package's
% \includegraphics command, latex and pdflatex will automatically select
% the appropriate figure format from those available.  This allows you
% to produce PS and PDF output from the same LaTeX source file.
%
% To generate a large format (e.g., 11"x17") PostScript copy for editing
% purposes, use
%
%  dvips -x 1467 -O -0.65in,0.85in -t tabloid advanced_example
%
% For further details and support, read the Users Manual, aiaa.pdf.

\documentclass[]{aiaa-tc}% insert '[draft]' option to show overfull boxes

 \usepackage{varioref}%  smart page, figure, table, and equation referencing
 \usepackage{wrapfig}%   wrap figures/tables in text (i.e., Di Vinci style)
 \usepackage{threeparttable}% tables with footnotes
 \usepackage{dcolumn}%   decimal-aligned tabular math columns
  \newcolumntype{d}{D{.}{.}{-1}}
 \usepackage{nomencl}%   nomenclature generation via makeindex
  \makeglossary
 \usepackage{amssymb,amsmath}
 \usepackage{subfigure}% subcaptions for subfigures
 \usepackage{subfigmat}% matrices of similar subfigures, aka small mulitples
 \usepackage{fancyvrb}%  extended verbatim environments
 \fvset{fontsize=\footnotesize,xleftmargin=2em}
 \usepackage{lettrine}%  dropped capital letter at beginning of paragraph
%  \usepackage[dvips]{dropping}% alternative dropped capital package
 \usepackage[colorlinks]{hyperref}%  hyperlinks [must be loaded after dropping]
 \usepackage{float}
 \usepackage{longtable,booktabs,tabularx}
 % \restylefloat{table}
 \usepackage{graphicx}
 \usepackage{caption}
 \usepackage{siunitx}
 \usepackage{multicol}
 \usepackage{indentfirst}
 \usepackage{environ}
 \usepackage[labelfont=bf]{caption}
 \usepackage{multirow}
 \usepackage{setspace}
%  \graphicspath{{./figs/}}
 \usepackage[sort, numbers]{natbib}

\usepackage{bm}
%\title{Conceptual Design of an Extremely Short Takeoff and Landing Aircraft (using GPkit/for Urban Air Mobility)}
\title{Feasibility Study of Short Takeoff and Landing Urban Air Mobility Vehicles using Geometric Programming}
 \author{
  Chris Courtin\thanks{Graduate Student, Aeronautics and Astronautics Engineering, MIT, 77 Mass Ave, Cambridge MA, 02139, AIAA Member.}, 
  Michael Burton\thanks{Graduate Student, Aeronautics and Astronautics Engineering, MIT, 77 Mass Ave, Cambridge MA, 02139, AIAA Student.}, 
  Patrick Butler\thanks{Graduate Student, Aeronautics and Astronautics Engineering, MIT, 77 Mass Ave, Cambridge MA, 02139, AIAA Student.}, 
  Alison Yu\thanks{Graduate Student, Aeronautics and Astronautics Engineering, MIT, 77 Mass Ave, Cambridge MA, 02139, AIAA Student.}, 
 Parker Vascik \thanks{Graduate Student, Aeronautics and Astronautics Engineering, MIT, 77 Mass Ave, Cambridge MA, 02139, AIAA Student.}, 
  John Hansman\thanks{T. Wilson Professor, Aeronautics and Astronautics Engineering, MIT, 77 Mass Ave, Cambridge MA, 02139, AIAA Member.} \\
  {\normalsize\itshape
   Massachusetts Institute of Technology, Cambridge, 02139, USA}\\
 }

 % Data used by 'handcarry' option
 \AIAApapernumber{YEAR-NUMBER}
 \AIAAconference{Conference Name, Date, and Location}
 \AIAAcopyright{\AIAAcopyrightD{YEAR}}

 % Define commands to assure consistent treatment throughout document
 \newcommand{\eqnref}[1]{(\ref{#1})}
 \newcommand{\class}[1]{\texttt{#1}}
 \newcommand{\package}[1]{\texttt{#1}}
 \newcommand{\file}[1]{\texttt{#1}}
 \newcommand{\BibTeX}{\textsc{Bib}\TeX}
 \usepackage{hyperref}
 \hypersetup{citecolor = blue}

\begin{document}

\graphicspath{{./figs/}}
\maketitle

\begin{abstract}
    The feasibility of an Urban Air Mobility (UAM) system that features electric Extremely Short Takeoff and Landing (ESTOL) vehicles is investigated.  An overview is given of the system constraints that must be incorporated into the design of the vehicle.  The system-wide advantages and limitations of ESTOL aircraft are discussed, for both near- and far-term system implementations.  A detailed vehicle sizing model is developed using geometric programming, a robust optimization framework.  This model is used to determine feasible boundaries on required runway size, vehicle range, and the sensitivity of the vehicle design to high-level mission parameters such as speed and number of passengers.  Key unique drivers of the vehicle design are identified.  The impact of distributed electric propulsion (DEP) is assessed.  Performance relative to a comparable Vertical Takeoff and Landing (VTOL) vehicle is analyzed, both with currently available technology and forecasted future technology.   The infrastructure requirements (runway size, approach paths, etc.) needed to support ESTOL operations are assessed according to current regulations.  Two major urban areas (Boston and Los Angeles) are presented as case studies to show where this infrastructure could be feasibly located.  Key challenges and risks to implementation are discussed.  


\end{abstract}

\section*{Nomenclature}

\begin{multicols}{2}
\small

\begin{tabbing}
  XXXXXXX \= \kill% this line sets tab stop
$A$ \> takeoff helper variable \\
$AR$ \> wing aspect ratio \\
$b$ \> wing span \\ % [ft] \\
$B$ \> takeoff helper variable \\
$c$ \> wing chord \\ %[m] \\
$C_D$ \> drag coefficient \\
$CDA$ \> area drag coefficient \\
$C_{D_g}$ \> ground drag coefficient \\
$c_{d_p}$ \> wing profile drag coefficient \\
$C_L$ \> lift coefficient \\
$C_{L_g}$ \> ground lift coefficient \\
$C_{L_{\mathrm{max}}}$ \> max lift coefficient \\
$D$ \> drag \\
$e$ \> span efficiency \\
$f_{\mathrm{struct}}$ \> fractional structural weight \\
$g$ \> gravitational constant \\
$L$ \> lift \\
$\mathcal{M}_{\mathrm{root}}$ \> root moment stress \\
$N$ \> deceleration factor \\
$P_{\mathrm{shaft-max}}$ \> max shaft power \\
$P_{\mathrm{spec}}$ \> specific motor power \\
$Re$ \> Reynolds number \\
$S$ \> wing area \\
$S_{\mathrm{land}}$ \> landing ground roll \\
$S_{\mathrm{runway}}$ \> runway distance \\
$S_{\mathrm{TO}}$ \> take off ground roll \\
$S_{y_{\mathrm{spar}}}$ \> spar section modulus \\
$t$ \> time \\
$T$ \> thrust \\
$V$ \> speed \\
$V_{\mathrm{stall}}$ \> stall speed \\
$W_{\mathrm{batt}}$ \> battery weight \\
$W_{\mathrm{fadd}}$ \> additional wing weight\\
$W_{\mathrm{motor}}$ \> motor weight \\
$W_{\mathrm{MTO}}$ \> max take off weight \\
$W_{\mathrm{pay}}$ \> payload weight \\
$W_{\mathrm{skin}}$ \> wing skin weight \\
$W_{\mathrm{spar}}$ \> wing spar weight \\
$W_{\mathrm{struct}}$ \> structural weight \\
$W_{\mathrm{wing}}$ \> wing weight \\
$\eta_{\mathrm{elec}}$ \> combined electric efficiency \\
$\eta_{\mathrm{prop}}$ \> propeller efficiency \\
$\mu$ \> rolling friction coefficient \\
$\rho$ \> air density \\
$\sigma_{\mathrm{CFRP}}$ \> carbon fiber allowable stress \\
 \end{tabbing}

\end{multicols}
% \printglossary% creates nomenclature section produced by MakeIndex

\section{Introduction}
Urban Air Mobility (UAM) is a broad concept that refers to a set of related operations and technologies that aim to provide on-demand intra-city and regional air transportation.  UAM concepts of some form or another have been around for at least the past fifty years.  Helicopters have been performing UAM-type missions since the 1960’s and have sufficient performance capability to meet nearly all payload, speed, and range requirements of the current UAM mission profile. However, historical helicopter public transportation networks by and large did not succeed and do not exist today due to the high costs of helicopter operation, the high noise generation of these vehicles, and the poor safety record of these aircraft.  Recently, advances in electric vehicle propulsion and key subsystem technologies has opened up the vehicle design space and enabled new approaches to mitigating some of these systemic challenges.  This has sparked renewed interest in the UAM concept and the development of UAM-specific vehicles.  Numerous legacy and emerging aircraft manufacturers are developing vehicles for this mission with novel configurations that leverage more-electric or all-electric vehicles and propulsion systems.  

There are a variety of system architectures that are being considered for UAM vehicles. Figure~\ref{f:ov} shows some of the most popular options, organized by number of propulsors and method of lift generation.  Of these architecture options, powered-lift distributed electric propulsion (DEP) and rotor-lift DEP are currently the two being most actively pursued. Powered-lift DEP is seen as being especially promising due to its VTOL capability and relatively efficient, high-speed cruise compared to a pure rotorcraft.  These vehicle concepts may have noise benefits compared to conventional helicopters.  These benefits arise in part from the lack of engine noise as well as the design freedom to vary tip speed and disk loading, both of which are key noise drivers. 
\begin{figure}
\begin{center}
	\includegraphics[width=.75\textwidth]{vehicle_architectures.png}
\caption{Overview of potential UAM aircraft system architectures}
\label{f:ov}
\end{center}
\end{figure}

[The UAM market is a high-risk proposition, in many different areas.  To maximize chances of successful implementation, risk should be reduced wherever possible.  From an assessment of the risk landscape, a STOL aircraft has advantages over a VTOL aircraft.  Feasibility questions remain, which this paper answers. ]

However, there are still significant challenges associated with these vehicle architectures.  First, the certification pathway for all-electric, fly-by-wire aircraft are not defined and represent a significant potential delay for system implementation in the United States. Secondly, it is unclear that noise can be reduced sufficiently with these configurations to the level that will allow widespread community acceptance.  Finally, the VTOL mission segment, reliance on lithium-ion batteries, and fly-by-wire features of these aircraft present significant safety issues that may be complex and costly to mitigate. 

One vehicle architecture which has not been widely considered for the UAM mission is the extremely short takeoff and land (ESTOL) aircraft, which is a fixed-wing aircraft capable of operating off a runway of 100-500ft.  By utilizing the wing throughout all phases of flight, ESTOL aircraft have performance advantages compared to VTOL aircraft of similar payload capacity.  If the runway can be made sufficiently short, the infrastructure differences between VTOL and ESTOL become relatively insignificant, and a useful network of runways can be constructed in close proximity to a major urban area.  ESTOL also has advantages over VTOL in noise and vehicle certification, which are two of the highest barriers to implementation of a function UAM system.  However, it is unclear how short a runway a vehicle could realistically be made to operate from, and how that requirement trades with speed, range, and payload.  

This research paper conducts an in-depth analysis of the feasibility of an UAM system that features ESTOL aircraft, considering both infrastructure and vehicle design constraints.  To conduct the vehicle design trade studies, a geometric programming (GP) vehicle sizing and performance model is developed to rapidly and holistically consider the large design space.  Ground infrastructure and approach path constraints are internalized in this model as design constraints on the aircraft.  This model is exercised to determine feasible bounds on runway operating length, and how vehicle range, speed, and payload capacity trade with ground infrastructure requirements.  
To assess the ground infrastructure requirements, the current FAA guidance for airport design and approach path layouts is used to create the nominal layout for a short takeoff and landing area (STOLA).  The feasibility of locating these STOLA near major urban areas is assessed by looking at case studies of two particular cities, Boston and Los Angeles.  The potential for ESTOL aircraft to act as a pathway to the development of large-scale UAM systems is reviewed from a technical, operational and business feasibility perspective. Furthermore, the complementary nature of ESTOL aircraft operations and potential future VTOL UAM operations are hypothesized. As part of this work, previous literature concerning small aircraft transportation system design ~\cite{Viken}, ~\cite{Holmes}, thin-haul transportation vehicles ~\cite{Harish},~\cite{Kreimeier},~\cite{Justin} �, VTOL aircraft ~\cite{Duffy}, and STOL aircraft is considered ~\cite{Antcliff},~\cite{SeeleyIV}. 

\section{Background}

\paragraph{UAM Risks}


\paragraph{Current Technology} ESTOL aircraft are not a new concept.  The Helio Courier, first built in the late 1940s, is one existing example, with demonstrated takeoff and landing distances from 100-300ft.  Highly experienced bush pilots are able to achieve landing distances on the order of single aircraft lengths.  Developing an aircraft that can operate of extremely short runways is clearly technologically feasible.  Nevertheless, ESTOL aircraft technology has not found wide adoption outside of the relatively small community of pilots who routinely fly to remote, relatively unimproved landing areas.   There are inherent challenges in taking existing ESTOL aircraft and using them for operations near major urban areas, which are a result of the design penalties that ESTOL capability drove on the vehicles.  
For example, current ESTOL vehicles achieve their short field performance through a combination of low wing loading, high power-to-weight, and extensive use of high-lift systems.  The low wing loading means that both maximum and best cruise speeds are fairly low.  Additionally, it makes the vehicles highly sensitive to gusts and turbulence, an important consideration for passenger operations.   The high-lift systems required are also complex (except with very low wing loadings), which adds a weight and cost penalty to these vehicles.  Depending on the type of powerplant type used, the need for high power at takeoff may also reduce the efficiency at the nominal cruising point.  More significantly, these high-power engines generated substantial noise on takeoff; a primary consideration when operating near urban areas. 

\paragraph{Key Enabling Technologies} The introduction of new electric aircraft technologies has the potential to change the paradigm of current ESTOL aircraft and make them practical for use in an UAM setting.   In the case of an all-electric aircraft, the reduction in noise through the use of batteries and an electric motor is one key area of improvement.  Another is the use of blown lift with distributed electric propulsion, in a configuration similar to the NASA X-57 Maxwell.  This has the ability to generate very high effective lift coefficients (especially on takeoff), with a reduction in the complexity of the high-lift system required and/or an increase in the climb path angle.  This technology allows in increase in the vehicle wing loading, which improves the cruising speed and passenger comfort.  
Additionally, since electric motors can be run above their maximum rated capacity for short periods of time, the weight penalties of installing a high-thrust system that is only needed on takeoff are reduced.  And since most of the motors would be shut down in cruising flight, the propulsion system can be designed to operate at or near peak efficiency throughout most of the mission.  When taken together, these key technologies enable the design of a practical ESTOL vehicle.  The range limitations come mostly from the use of batteries and their poor specific power relative to hydrocarbon fuels; replacing them with a hybrid-electric system could extend the range or reduce vehicle takeoff weight, with associated tradeoffs on noise, emissions, and direct operating cost.  
 


\section{Market Analysis}

[Assert our market case for stol aircraft; 50-100nmi]

\section{Vehicle Feasibility}

A sizing study using Geometric Programming optimization was performed to understand how vehicle performance and design would be effected by short take offs and landings. 
This section describes the assumptions and equations used in the optimization model for vehicle size, cruise performance, and takeoff and landing distances.

Geometric programming was selected as a means of evaluating this trade space because of its speed and reliability.  
Geometric programming is a special type of convex, non-linear optimzation.\cite{gp}
Because it is convex, even GPs with thousands of variables can be solved quickly.\cite{gp}
Additionally, recent research has shown that GPs can be used to evaulate aircraft design trade spaces.\cite{burton_solar_2017}\cite{gpkit}


\subsection{Vehicle Model}

It is assumed that the aircraft is completely electric, replying on battery power for powered flight. 
The aircraft weight is comprised of the battery, payload, wing, motor, and structural weight,

\begin{equation}
    W_{\mathrm{MTO}} \geq W_{\mathrm{batt}} + N_{\mathrm{pax}}W_{\mathrm{pax}} + W_{\mathrm{wing}} + W_{\mathrm{motor}} + W_{\mathrm{struct}}
\end{equation}

where the motor, passenger, and structural weights are

\begin{align}
    W_{\mathrm{motor}} &\geq \frac{P_{\mathrm{shaft-max}}}{P_{\mathrm{spec}}} \\
    W_{\mathrm{pax}} &= 195 \mathrm{[lbf]} \\
    W_{\mathrm{struct}} &\geq W_{\mathrm{MTO}}f_{\mathrm{struct}}.
\end{align}

The battery weight is constrained by the range of the aircraft

\begin{equation}
    R \leq \frac{h_{\mathrm{batt}} W_{\mathrm{batt}} \eta_{\mathrm{elec}} V}{gP_{\mathrm{shaft}}}
\end{equation}

where the shaft power is 

\begin{equation}
    P_{\mathrm{shaft}} \geq \frac{TV}{\eta_{\mathrm{prop}}}
\end{equation}

The aircraft is assumed to be in steady level flight during cruise. 

The wing weight is composed of the skin, main spar and additional components

\begin{equation}
    W_{\mathrm{wing}} \geq W_{\mathrm{skin}} + W_{\mathrm{spar}} + W_{\mathrm{fadd}}
\end{equation}

The skin and structural elements are assumed to be carbon fiber.  
The wing spar configuration is a cap spar with unidirectional carbon fiber caps wrapped in a shear web as shown in Figure~\ref{f:capspar}.  

\begin{figure}[h!]
	\begin{center}
	\includegraphics[width=0.9\textwidth]{capspar.pdf}
    \caption{\textbf{Cross sectional view of a cap spar.}}
	\label{f:capspar}
	\end{center}
\end{figure}

The spar dimensions are sized such that the material stresses are not exceeded under a 3.5 g-load,

\begin{equation}
    \sigma_{\mathrm{CFRP}} \geq \frac{\mathcal{M}_{\mathrm{root}}}{S_{y_{\mathrm{spar}}}}
\end{equation}

The root wing moment $\mathcal{M}_{\mathrm{root}}$, is calculated assuming a distributed load along the wing span that scales with the local chord.\cite{bending}
A constant tapered wing is assumed.  
This wing sizing model leverages the GP wing sizing model used by Burton and Hoburg.\cite{burton_solar_2017} 

A simple drag model is used for the aircraft, 

\begin{equation}
    C_D \geq CDA + c_{d_p} + \frac{C_L^2}{\pi e AR}.
\end{equation}

where the profile drag coefficient $c_{d_p}(C_L, Re)$, is calculated from a representative wing polar. 
The combined drag and wing loading models allow the aspect ratio to be optimized, trading structural integrity with aerodynamic performance. 

\subsection{Takeoff and Landing Models}

The takeoff model was adapted from Raymer's takeoff equations to fit a GP compatible form.  Using equations of motion the takeoff state can be expressed

\begin{equation}
    T - D - \mu(W_{\mathrm{MTO}} - L) = \frac{W_{\mathrm{MTO}}}{g} \frac{dV}{dt}.
\end{equation}

This can be simplified to 
\begin{align}
    \frac{dV}{dt} &= g \left( \frac{T}{W_{\mathrm{MTO}}} - \mu \right) - \frac{g}{W_{\mathrm{MTO}}} \left( \frac{1}{2} \rho S V^2 (C_{D_g} - \mu C_{L_g})\right) \\
    \label{e:todiff}
    \frac{dt}{dV} &= \frac{1}{A-BV^2}
\end{align}

The takeoff ground run distance can then be expressed by taking the integral of Equation~\ref{e:todiff} to achieve

\begin{equation}
    \label{e:to}
    S_{\mathrm{TO}} = \frac{1}{2B} \ln{\frac{A}{A-BV^2}} 
\end{equation}

The natural log function can be approximated to make Equation~\ref{e:to} GP-comptible by 

\begin{equation}
    \ln{\frac{A}{A-BV^2}} \approx \num{5.6e-4} A^{-6.04} (BV^2)^{6.04} + 1.0 A^{-0.001} (BV^2)^{0.001} + \num{7.5e-4} A^{-1.276} (BV^2)^{1.275}
\end{equation}

with an average log error of 0.06\%.  The terms $A$, and $B$, are constrained by

\begin{align}
    \frac{T}{W_{\mathrm{MTO}}} &\geq \frac{A}{g} + \mu \\
    B &\geq \frac{g}{W_{\mathrm{MTO}}} \frac{1}{2} \rho S C_{D_g}
\end{align}

where the $\mu C_{L_g}$ term is neglected as a conservative approximation for $B$ to preserve GP-compatibility. 

The landing ground roll distance is calculated using conservation of energy, with the primary design variable being the loading deceleration factor, $N$.
This constraint will drive the wing loading down. 

\begin{equation}
    S_{\mathrm{land}} \geq \frac{1}{2} \frac{V^2}{Ng} 
\end{equation}

where $N=1$ corresponds to a 1-g deceleration. 
The deceleration factor is a function of the technologies used to stop the aircraft and include, but are not limited to: breaks, reverse thrust from electric motors, and drag.  
To understand how the g-loading constant varies with different amounts of reverse thrust, the ground roll and deceleration factor are calculate for the X-57 as an example case. 
Table~\ref{t:landingdecel} shows the deceleration loading factor for different amounts of reverse thrust. 

\begin{table}[H]
    \centering
    \caption{Landing Case for the X-57}
    \label{t:landingdecel}
    \begin{tabular}{lcc}
    \toprule
    \toprule
                                    & Ground          & Deceleration \\ 
                                    & Roll Distance   & Factor ($N$)\\ \hline
    Brakes only (dry)               &  925 [ft]       & 0.37  \\
    Brakes + 10\% reverse thrust    &  850 [ft]       & 0.4   \\
    Brakes + 50\% reverse thrust    &  625 [ft]       & 0.55  \\
    Brakes + 100\% reverse thrust   &  425 [ft]       & 0.73  \\
    \bottomrule
\end{tabular}
\end{table}

For both the landing and takeoff constraints it is assumed that the velocity has a 20\% margin on the stall velocity,

\begin{equation}
    V = 1.2V_{\mathrm{stall}} = \sqrt{\frac{2W_{\mathrm{MTO}}}{\rho S C_{L_{\mathrm{max}}}}}.
\end{equation}

It is assumed that the the max lift coefficient is different for landing and takeoff.
Another 40\% margin is placed on the ground roll distance to determine runway length

\begin{align}
    S_{\mathrm{runway}} &\geq 1.4S_{\mathrm{TO}} \\
    S_{\mathrm{runway}} &\geq 1.4S_{\mathrm{land}} 
\end{align}

\subsection{Vehicle Trade Studies}

Using the geometric programming model of a STOL aircraft perviously described, tradeoffs between runway length and vehicle performance were evaulated.  The models consists of a 105 free variables and was solved in 0.114 seconds with an objective function to minimize weight, $\min{(W_{\mathrm{MTO}})}$. Key parameters are defined in Table~\ref{t:params} and important solution variables are shown in Table~\ref{t:vars}.

\begin{multicols}{2}

\begin{table}[H]
    \centering
    \caption{Design Parameters}
    \label{t:params}
    \begin{tabular}{l c}
    \toprule
    \toprule
    Parameter                                   & Value         \\ \hline
    $S_{\mathrm{runway}}$                       & 300 [ft]      \\
    $\eta_{\mathrm{elec}}$                      & 0.9           \\
    $h_{\mathrm{batt}}$                         & 210 [Whr/kg]  \\
    $P_{\mathrm{spec}}$                         & 0.7136 [kW/N] \\
    $R$                                         & 100 [nmi]     \\
    $V_{\mathrm{min}}$                          & 100 [kts]     \\
    $C_{L_{\mathrm{max}}}$ (Landing)            & 3.5           \\
    $C_{L_{\mathrm{max}}}$ (TO)                 & 4.0           \\
    $N$                                         & 0.3g          \\
    $\eta_{\mathrm{prop}}$                      & 0.8           \\
    \bottomrule
\end{tabular}
\end{table}

\begin{table}[H]
    \centering
    \caption{Design Variables}
    \label{t:vars}
    \begin{tabular}{l c}
    \toprule
    \toprule
    Parameter                                   & Value \\ \hline
    $W_{\mathrm{MTO}}$                          & 1496 [lbf] \\
    $W_{\mathrm{batt}}$                         & 278 [lbf] \\
    $W_{\mathrm{wing}}$                         & 53 [lbf] \\
    $AR$                                        & 9 \\
    $b$                                         & 22.3 [ft] \\
    $V_{\mathrm{stall}}$                        & 48 [kts] \\
    $(W/S)$                                     & 27 [lbf/ft$^2$] \\
    \bottomrule
\end{tabular}
\end{table}

\end{multicols}

To understand how passenger and runway requirements affect vehicle weight, the GP model was solved 30 times in 3.46 seconds.  
The results are shown in Figure~\ref{f:sw_mtow}, each point on the graph corresponding to a unique optimization solution or vehicle size.  
From this study it is observed that for runway lengths shorter than 250 ft are near infeasible for this set of parameters.  
It is also oberserved that the runway length is fairly insenstive to number of passengers.  

\begin{figure}[h!]
 \begin{subfigmatrix}{2}% number of columns
     \subfigure[\label{f:sw_mtow}Contours of number of passengers]{\includegraphics[]{sw_mtow.pdf}}
     \subfigure[\label{f:sw_mtowsens}Sensitivity to landing constraints]{\includegraphics[]{sw_mtowsens.pdf}}
 \end{subfigmatrix}
    \caption{\textbf{Trade space of aircraft weight, number of passengers and runway length.}}
 \label{f:sw_mt}
\end{figure}

By looking at the sensitivity to the lift coefficient, shown in figure~\ref{f:sw_mtowsens}, on landing for the same trade study shown in figure~\ref{f:sw_mtow}, it can be determined whether the aircraft is constrained by the landing constraints or the take off constraints. 
The sensitivity to a variable in a geometric program is defined as the percentage change in the objective function for a 1\% change in that variable's value.  
For this study, either the landing constraints or the take off constraints will be active or driving the size of the vehicle.  
Thus, if the sensitivity to the lift coefficient on landing is zero, then the landing constraints are not active. 
Conversly, if the sensitivity is non-zero then those constraints are active and the vehicle size is landing constrained. 

Another way to view this trade space is to understand how runway length and range are effected by different minimum speed requirements.  
Figure~\ref{f:minspeed} shows plots of minimum cruise speed vs max take off weight for different contours of runway length and range requirements.  
This trade study was done for 5 passengers.  
These plots show that short runway and longer range requirements are possible but require slower cruise speeds.  
Because power scales with velocity cubed, slower speed reduces the required power and therefore battery weight, which improves the whole system. 
Note that in both plots shown in figure~\ref{f:minspeed}, lowering the minimum speed does not always improve the vehicle weight.  
A flat curve indicates that the aircraft is not constrained by the minimum cruise speed because of the optimum cruise speed for that set of requirements is faster than the minimum cruise speed. 

\begin{figure}[h!]
 \begin{subfigmatrix}{2}% number of columns
     \subfigure[\label{f:vweightR}Contours of range]{\includegraphics[]{vweightR.pdf}}
     \subfigure[\label{f:vweightS}Contours of runway length]{\includegraphics[]{vweightS.pdf}}
 \end{subfigmatrix}
 \caption{\textbf{Trade study between requirements of runway length, minimum speed and range.}}
 \label{f:rangetod}
\end{figure}

\subsection{Advanced Technology Trade Studies}

The previous section showed fundamental trade studies and trends for how runway length varies with performance.  
It is also possible to shorten runway length through advanced technology. 
As disscussed previously, a number of technologies could help reduce the required runway length including power bursts from electric motors, reverse thrust on landing, advanced flight controls, and improved battery technology.  
The effect of these technology advances on required runway length can be observed by changing a few parameters from the baseline case and resolving the optimization model.
Table~\ref{t:techparams} compares the baseline parameters to the advanced technology parameters that were assumed in the optimization model. 

\begin{table}[H]
    \centering
    \caption{Advanced Technology Parameter Assumptions}
    \label{t:techparams}
    \begin{tabular}{l c c}
    \toprule
    \toprule
    Parameter                         & Baseline Value  & Advanced Value \\ \hline
    $h_{\mathrm{batt}}$               & 210 [Whr/kg]    & 300 [Whr/kg]   \\
    $P_{\mathrm{spec}}$               & 0.7136 [kW/N]   & 0.571 [kW/N]   \\
    $C_{L_{\mathrm{max}}}$ (Landing)  & 3.5             & 4.5            \\
    $C_{L_{\mathrm{max}}}$ (TO)       & 4.0             & 5.0            \\
    $N$                               & 0.3             & 0.5            \\
    Runway margin                     & 40\%            & 20\%           \\
    Stall speed margin                & 30\%            & 10\%           \\
    \bottomrule
\end{tabular}
\end{table}

The motor power to weight ratio was lowered to take credit for the electric motor power burst during take off and is 80\% of the baseline value. 
The maximum lift coefficient during landing and takeoff were increased to match values predicted by NASA.
The landing deceleration factor was increased to take credit for use of reverse thrust on landing.
The lower margins on the runway length and stall speed are assuming advanced flight controls that require less margin. 
Realistically achieving all of these advances in a STOL aircraft design is unlikely.  
However, understanding the extremes between the baseline, conservative case and a more optimistic case is useful in determining a feasible vehicle design and vehicle requirements. 
Figure~\ref{f:sw_mtt} shows the same trade study as figure~\ref{f:sw_mt}, but with the updated parameter values shown in Table~\ref{t:techparams}.

\begin{figure}[h!]
 \begin{subfigmatrix}{2}% number of columns
     \subfigure[\label{f:sw_mtowt}Contours of number of passengers]{\includegraphics[]{sw_mtowt.pdf}}
     \subfigure[\label{f:sw_mtowtsens}Sensitivity to landing constraints]{\includegraphics[]{sw_mtowtsens.pdf}}
 \end{subfigmatrix}
    \caption{\textbf{Trade space of aircraft weight, number of passengers and runway length for advanced technology assumptions.}}
 \label{f:sw_mtt}
\end{figure}

As observed in figure~\ref{f:sw_mtowt}, the advanced techonology assumptions allow for a much shorter runway than the baseline case showing that runways below even 100 ft might be possible. 
To understand which parameters have the largest effect on this trade study, each parameter can be varied one at a time from the baseline case. 
Figure~\ref{f:tech} show variations on the 5 passenger contour from figure~\ref{f:sw_mtow}. 

\begin{figure}[h!]
 \begin{subfigmatrix}{3}% number of columns
     \subfigure[\label{f:smtow_clmax}Contours of $C_{L_{\mathrm{max}}}$]{\includegraphics[]{smtow_clmax.pdf}}
     \subfigure[\label{f:smtow_gl}Contours of deceleration factor]{\includegraphics[]{smtow_gl.pdf}}
     \subfigure[\label{f:smtow_hbatt}Contours of battery specific energy]{\includegraphics[]{smtow_hbatt.pdf}}
 \end{subfigmatrix}
    \caption{\textbf{Trade space of aircraft weight, number of passengers and runway length for advanced technology assumptions.}}
 \label{f:tech}
\end{figure}

Note that increasing the maximum lift coefficient or the deceleration factor has no effect for higher runway lengths.  
This is because at higher runway lengths the size of the aicraft is constained by the range requirement but not the runway requirement. 
Increasing the battery specific energy however, is always beneficial because that lowers the battery weight which improves the whole system. 

\section{Infrastructure}
\begin{figure}[!htbp]
\centering
	\includegraphics[width=0.5\textwidth]{STOLport_300_nominal.eps}
\caption{The layout of a notional short takeoff and landing area}
\label{fig:stolport}

\end{figure}

Figure ~\ref{fig:stolport} shows the layout of a notional STOLA for a 300ft runway length, taking into account current FAA guidance on runway clearance and centerline separation for runways and taxiways.  This assumes that the vehicles will have an approach reference speed of less than 50 kts.  This layout is notional and is affected by vehicle approach speed and wingspan. 

A key part of the feasibility of an UAM concept is dual-use takeoff and landing areas (TOLA), whether for ESTOL or VTOL vehicles.   Due to the high value and scarcity of undeveloped urban real-estate, single-use UAM infrastructure is likely to be cost-prohibitive.  Considerations for various types of STOLAs are shown in Figure~\ref{fig:stola_key}, which has notional layouts for STOLAs over roads, railways, barges, and buildings. 
\begin{figure}[!htbp]
\centering
	\includegraphics[width=1\textwidth]{STOLport_key.jpg}
\caption{Notional STOLport layouts for various locations}
\label{fig:stola_key}
\end{figure}
Future work will lay out more fully the design constraints of a notional STOLA, how the throughput is constrained by various systemic factors (vehicle separation, charging time/infrastructure, ground access, passenger loading times), and how that is reflected in the vehicle design. 




\section{Conclusion}

\bibliography{biblibrary}
\bibliographystyle{aiaa}

\end{document}

